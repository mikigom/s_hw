\documentclass[11pt,a4paper]{article}
\usepackage{times}
\usepackage{mathtools}
\usepackage{url}

\begin{document}

\title{Statistics 315B Homework 2}
\author{Spring 2017}

\maketitle

\renewcommand{\thesubsection}{\thesection.\alph{subsection}}

\section{}

This answer is referred to in \cite{ho1998random}'s \textit{random subsampling method} and \cite{li2009weighted}.

\subsection{Advantages}
\begin{itemize}
    \item Free from curse of dimensionality
    \item The algorithm is scalable.(A parallel algorithm)
    \item It does not get hit by bootstrapping or bagging in generalization.
    \item Effectively eliminating the feature redundancy inherent in the data.
    \item Effectively reducing the correlation between estimators.
\end{itemize}

\subsection{Disadvantages}
\begin{itemize}
    \item The original random subspace method did not emphasize the importance of selecting a base learner that should generalize well.
\end{itemize}

\subsection{}
Bagging or bootstrapping is possible. However, some papers and work\cite{shlien1990multiple}\cite{shlien1992nonparametric} have also introduced parametric or nonparametric heuristic procedures or manual interventions.

\section{}


\bibliographystyle{plain}
\bibliography{critique}

\end{document}
